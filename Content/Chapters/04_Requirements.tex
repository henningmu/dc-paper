%!TEX root = ../../Paper.tex

\chapter{Requirements for Benchmarking Distributed Systems}
\label{cha:requirements}
The previous chapter illuminated existing approaches of benchmarking in general and of benchmarking distributed systems and databases in specific. The goal of this chapter is to aggregate a set of requirements needed to benchmark distributed systems. This set shall not only cover aspects of the existing approaches but be enhanced with specialized requirements for the given use case.

The benchmarks defined by the \ac{TPC} have a clear and bold focus on an application driven workload mix inspired by real-world applications. They define minimum percentages for every possible transaction or user request. This can be a good approach when an application is analysed which already has users or a known usage profile which can be used as a prototype for the workload mix. Sometimes benchmarks are conducted to assist in decision making process which technology shall be applied for a new technology. In this case no usage profile is available and it is important to have the ability to define a custom workload. This should be done similar to the YCSB where distributions for specific operations are declared. Distributions in contrast to random values have the big advantage to simulate specific usages for example reading from data hotspots or updating the most current entries in the database.

Additionally to workload mix the measured performance metrics have to be taken into account. The \ac{TPC} measures business transactions and the business throughput hence measuring fully completed user request. Another approach is to measure the throughput as request per seconds. In the context of the distributed an interesting metric to observe is the amount of servers needed to achieve a constant throughput.

\begin{itemize}
  \item when measuring amount of servers the price for the infrastructure could be calculated at every time
  \item utilization of the servers has to be monitored to explain spikes and reject measurement errors
  \item easily extendible (without changes of -> generic config file)
  \item statistic module is optional important is that the data is persisted -> there are enough tools to analyse: SAP Lumira, Excel, R, ...
  \item adapt Rain by splitting generation from execution to repeat same workload over and over
\end{itemize}
