%!TEX root = ../../Paper.tex

\chapter{Introduction}
\label{cha:introduction}

\section{Motivation and Problem Statement}
\label{sec:motivation-and-problem-statement}
In the recent past a shift to big data emerged. IBM discovered that every day $2.5$ quintillion bytes of new data are created and $90\;\%$ of the world's data has been created in the past two years \cite{ibm.2014}. With more and more applications being deployed to the cloud new technologies are being utilized. A shift from relational (SQL) databases to non-relational (NoSQL) databases is observable. These new alternatives to the relational databases are capable of not only handling the massive amounts of data, but also scale seamlessly and stand out with a high availability. This makes them a perfect fit for Web 2.0 applications which are faced with high variations in the usage hence in the load on their servers.

An example is the famous on-demand video streaming provider Netflix \cite{netflix.2014}. Netflix states that they grew so fast that they did not manage to scale their own data centres accordingly, at the peak times they are responsible for $29.7\;\%$ of the downstream traffic in the US \cite[1]{adhikari.2012}. They rely on Amazon's public cloud for lower latency, avoid own data centres, high robustness and availability and a more agile development team \cite[24]{netflix.2011}.

For enterprises and developers alike, it is hard to decide which NoSQL database to utilize. There are more than 60 different databases grouped in the 4 categories: Column Stores, Document Stores, Key-Value Stores and Graph Databases \cite[1 - 2]{tudorica.2011}. Even if developers are being able to decide which category is best suited for their specific use case it is hard to choose the right database. When it comes to choosing which database to use decisions are based upon preferences or affections because performance data for real-world use cases are missing.

\section{Objectives}
\label{sec:objectives}
As stated in the previous chapter it is a hard task to choose the right NoSQL database for a given task. This is caused by two main problems: first there are too many alternatives to choose from and second performance data for specific use cases are missing to compare the different databases.
This paper has the objective to define a set of requirements which a benchmark designed for distributed databases should implement. By outlining a standard set of requirements it gets more easily to compare different databases in the same situation.

There are already some existing approaches to benchmark distributed systems. The problem they share is that they are not generic enough and too hard to implement and run. Existing benchmarks focus on specific applications or are only described in general and not yet implemented and ready to run. This paper tackles these problems by describing what has to be implemented to fulfil the set of requirements.


\section{Structure}
\label{sec:structure}
This paper is structured in five chapters. The remaining part is structured as follows.

The second chapter explains the theoretical background. It covers the foundations about performance metrics, workload types, distributed systems and NoSQL.

In chapter three existing approaches to benchmark distributed systems and databases are presented. The fourth chapter assembles a set of requirements a benchmark for distributed databases should implement.

The last chapter concludes the paper by summarizing the results and giving an outlook for future works.
