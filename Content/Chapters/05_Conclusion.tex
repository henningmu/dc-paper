%!TEX root = ../../Paper.tex

\chapter{Conclusion}
\label{cha:conclusion}
This paper has focussed on the objective to provide a set of requirements to benchmark distributed databases. The prevalent problems of too many databases with insufficient performance data for specific use cases motivated this paper. After analysing the existing approaches to benchmark distributed systems and identifying core features for a benchmark the set of requirements was compiled (cf.~Table~\ref{tab:requirements}). The existing approaches highlighted important factors when benchmarking distributed systems, however they all focused on single use cases or scenarios and did not consider the bigger picture.

The set of requirements shall be used as a foundation when building a benchmark framework capable of measuring distributed databases. Future works could analyse whether it is necessary to implement the benchmark from scratch or if existing approaches could be easily extended (e.g. Rain or YCSB). When implementing the benchmark an analysis of technologies has to be performed in order to figure out which frameworks and patterns are best suited for the use case. Additionally, it could be explored, what common workload mixes exist. These should be defined and integrated in the implementation.

All in all the paper achieved its goals by defining a generic set of requirements usable to benchmark distributed databases. An outlook for future works has been outlined.
